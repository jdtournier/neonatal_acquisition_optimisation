\documentclass{article}

\usepackage[pdftex]{color,graphicx}
%\usepackage{lmodern}
%\usepackage[T1]{fontenc}
%\usepackage{textcomp}
\usepackage[margin=2cm]{geometry}
%\usepackage[space]{grffile}
%\usepackage{adjustbox}
\usepackage{array}
\usepackage{amsmath}
\usepackage{multirow}
\usepackage{booktabs} % For \toprule, \midrule and \bottomrule
\usepackage{siunitx} % Formats the units and values
\usepackage{pgfplotstable} % Generates table from .csv

% Setup siunitx:
\sisetup{
  round-mode          = places, 
  round-precision     = 3, 
}

\newcommand{\nDW}{n_{\textrm{DW}}}
\newcommand{\nvox}{n_{\textrm{vox}}}
\newcommand{\mms}{\textrm{mm}^2/\textrm{s}}
\newcommand{\cnr}{\textrm{CNR}}
\DeclareMathOperator*{\argmin}{arg\,min}
\DeclareMathOperator{\Lagr}{\mathcal{L}}


\newcommand{\imagingassumptions}{This is calculated assuming a total of $\nDW = 400$ DW directions, and SNR$_{b=0} = 15$.}

\newcommand{\captionbval}[2]{
\caption{Optimal \textbf{$b$-values} for a #1-shell acquisition (excluding $b=0$), targeting #2 
coefficients, neglecting $T_2$ effects, per subject, and for all subjects combined.}
}
\newcommand{\captionnDW}[2]{
\caption{Optimal \textbf{number of directions} per shell for a #1-shell acquisition, 
targeting #2 coefficients, neglecting $T_2$ effects, per subject, and for all subjects combined.
\imagingassumptions }
}
\newcommand{\captionCNR}[2]{
\caption{Optimal \textbf{contrast to noise ratio} per coefficient for a #1-shell acquisition,
targetting #2 coefficients, neglecting $T_2$ effects, per subject, and for all subjects combined.
\imagingassumptions }
}

\newcommand{\captionbvalTE}[3]{
\caption{Optimal \textbf{$b$-values} for a #1-shell acquisition (excluding $b=0$), targeting #2 
coefficients, assuming $T_2 = #3$ ms, per subject, and for all subjects combined.}
}
\newcommand{\captionnDWTE}[3]{
\caption{Optimal \textbf{number of directions} per shell for a #1-shell acquisition, 
targeting #2 coefficients, assuming $T_2 = #3$ ms, per subject, and for all subjects combined.
\imagingassumptions }
}
\newcommand{\captionCNRTE}[3]{
\caption{Optimal \textbf{contrast to noise ratio} per coefficient for a #1-shell acquisition,
targetting #2 coefficients, assuming $T_2 = #3$ ms, per subject, and for all subjects combined.
\imagingassumptions }
}

\author{J-D.~Tournier}
\title{optimisation of dHCP diffusion acquisition protocol}

\begin{document}
\maketitle

\section{Overview}

Basic idea: the DW signal within each voxel is a linear combination of tissue-specific
signatures. These tissue signatures may have orientational features (notably
for white matter), but what is critical for the determination of the `optimal'
multi-shell scheme is that their DW signal averaged over all orientations per
shell (i.e.  the `DC' term) have distinct features as a function of $b$-value. 

The linear combination approach is generally assumed valid in the limit of
negligible exchange between tissue types - this is used in many complex models
of DW signal in white matter. It's completely valid when the different
components correspond to physically distant regions of tissue, for example when
two tracts pass close to each other (leading to partial volume averaging),
since the water molecules will not be able to diffuse between compartments
during the applied diffusion time. It is also probably justified even when different
compartments co-exist given that the pre-exchange lifetime of water molecules
in cells is typically estimated at between 0.5 to 1s, which is long compared to
the $\sim$50ms diffusion times typically used. 

The focus on the isotropic part of the signal, ignoring orientational features,
can also be justified as follows. First, in the absence of other information, any
orientation feature in the `intrinsic' DW signal cannot be disentangled from
the orientation distribution function (ODF) for that tissue in that voxel - in
other words, orientation features may change simply due to a rotation of the
tissue, or the presence of multiple orientations of it within the same voxel
(cf. crossing fibres). On the other hand, the DC term of that signal is
rotationally invariant, and directly proportional to the amount of
corresponding tissue. Second, while focusing on the isotropic part may discard
usable information (in terms of teasing apart the various compartments), this
implies that the contrast-to-noise measures estimated by the procedure
described below are conservative -- it may be possible to improve on the values
given, but they nonetheless constitute a sane lower bound. 

\subsection{General approach}

Based on the above, the idea is to factorise the DW signal into 3 parts: 
\begin{itemize}
\item a set of intrinsic tissue-specific {\em responses};
\item their corresponding {\em effect sizes} (i.e. how much they contribute to the signal for a `unit' response);
\item a vector of voxel-wise {\em weights} giving the amount of each response per voxel.
\end{itemize}

Given such a factorisation, and a set of $b$-values to use, it is possible to
identify the optimal number of directions $\left\{ d_i \right\}$ per $b$-value (as a fraction of the
total number acquired), such that the contrast-to-noise ratio (CNR) of the
estimated weights is maximised. 

Identifying the optimal set of $b$-values then becomes a matter of optimising
for maximum CNR (given optimised $\left\{ d_i \right\}$ as above).

\subsection{Signal factorisation}

There are a variety of ways to achieve this, with the most intuitive being the
non-negative matrix factorisation approach; unfortunately this turns out to
have stability issues. A more tried-and-tested matrix decomposition is the
singular value decomposition (SVD), which actually has some nice properties --
not least of which is the fact it does directly break down the signal as above. 

Given a $\nvox \times n_b$ matrix of data values $D$, we use the
slim SVD to provide matrices $W$, $E$ and $H$ such that $D = W E H^T$, where:
\begin{itemize}
\item $W$ is the $\nvox \times n_c$ matrix of weights for each coefficient, per voxel;
\item $E$ is the $n_c \times n_c$ diagonal matrix of effect sizes for each
coefficient -- also referred to as $\epsilon_i$ throughout this document;
\item $H$ is the $n_b \times n_c$ orthogonal matrix of response function,
giving the expected signal for a unit response at each $b$-value.
\end{itemize}
In the above, $n_b$ corresponds to the number of $b$-values in the acquisition
(including the $b=0$), $n_c$ to the number of response functions in the data,
and $\nvox$ to the number of voxels within the image mask. 

To achieve the above, the matrices $D = USV^T$ of the slim SVD needs to be
modified, since the columns of $U$ are normalised to unity, whereas we need the
elements of $W$ to have unit root-mean-square value -- this is necessary to
ensure that the effect sizes are meaningful. To achieve this, the SVD
output is modified as follows: 
\begin{eqnarray}
W & = & U \times \sqrt{\nvox} \\
E & = & S / \sqrt{\nvox} \\
H & = & V
\end{eqnarray}


\subsubsection{Interpolation over the $b$-value domain}
In order to enable a more fine-grained optimisation of the $b$-values,
and allow the estimation of contrast to noise at $b$-values other than those
specified, a suitable interpolation approach is required. There are many ways
to achieve this; here we chose to use the following: 
\begin{enumerate}
\item the data matrix $D$ is interpolated along the $b$-value dimension using
the Piecewise Cubic Hermite Interpolating Polynomial (PCHIP) interpolation
routine provided in Matlab, to a final resolution $\Delta_b = 100 \mms$, to
produce the $\nvox \times n_{b'}$ matrix $D'$, with $n_{b'} = 1 + b_{\textrm{max}} / \Delta_b$ 
($n_{b'} = 41$ in this case). The PCHIP algorithm was chosen based on its
guaranteed monotonicity when applied to monotonic data (as is the case here).
\item the slim SVD is performed on the interpolated matrix $D'$, but only the
sub-matrices corresponding to the first $n_b$ singular values are retained. The
matrix $H_{\textrm{upsampled}}$ is therefore now $n_c \times n_{b'}$, with the size of the other
matrices remaining unaffected.
\item responses $H$ for a given set of arbitrary $b$-values can then be derived
using a further PCHIP interpolation step on the response matrix $H_{\textrm{upsampled}}$.
\end{enumerate} 


\subsection{Noise Propagation}
Given a problem $c = M s$, the variance-covariance matrix $\Sigma_c$ on the coefficients $c$ is given by:
\begin{equation}
\label{noise_prop}
\Sigma_c = M \Sigma_s M^T 
\end{equation}
where $\Sigma_s$ is the variance-covariance matrix on the measured data $s$.
The diagonal elements of $\Sigma_c$ correspond to the variances 
$\left\{ \sigma_i^2 \right\}$ of the estimated coefficients. The CNR of the
individual coefficients can then be derived given an estimate of the
corresponding effect size $\epsilon_i$:
\begin{eqnarray}
\cnr_i & = & \frac{\epsilon_i}{\sigma_i} \\
& = & \frac{E(i,i)}{\sqrt{\Sigma_c(i,i)}}
\end{eqnarray}

Assuming the data are independent and identically distributed (i.i.d.),
$\Sigma_s = \sigma_s^2 I$, where $I$ is the identity matrix, and $\sigma_s$ is
the standard deviation of the imaging noise. In our problem, the matrix $M$
corresponds to $H^\dagger$, the Moore-Penrose Pseudo-inverse of $H$, the matrix
of responses evaluated over the desired set of $n_b$ $b$-values.

When multiple measurements of the same data point are available, the result is
essentially to reduce the variance of the measurement by the corresponding
factor -- i.e. $\Sigma_s(i,i) = \sigma_s^2 / d_i$, where $d_i$ is the number of
measurements performed for data point $i$. In the context of the present
analysis, repeated measurements at the same $b$-value will typically be taken
along different DW directions, and so will not correspond. However, provided
the total number of measurements taken at each $b$-value exceeds the minimum
necessary for aliasing-free reconstruction, the variance of the DC term
estimated for that $b$-value is in fact characterised by the same relationship
(Tournier et al., NMR Biomed. 2013). In other words, assuming the sampling of
the orientation domain is uniform and sufficiently dense, the variance of the
mean DW signal at that $b$-value is also inversely proportional to the
number of measurements performed at that $b$-value. 

For the present study, a spherical harmonic order $l = 6$ was shown to be
sufficient to fully characterise the DW signal at high $b$-values, implying
that a set of 28 directions was sufficient to ensure sufficiently dense
sampling. This number is reduced to order $l=4$ (15 DW directions) for lower
$b$-values, and eventually through $l=2$ (6 measurements) to $l=0$ (1 
measurement) at $b=0$. Hence, the above argument holds provided the number of 
measurements at each $b$-value is greater than those numbers. 



\subsection{Determining the optimum number of DW directions}

Given an acquisition comprising $n_b$ shells (including $b=0$), $n_c = n_b$ 
coefficients can be estimated. To determine the best number of DW directions to
acquire per shell, a good cost function to minimise is the sum of their
squared coefficients of variation:
\begin{eqnarray}
\label{efficiency}
\Psi & = & \sum_i^{n_b} \left( \frac{\sigma_i}{\epsilon_i} \right)^2 \\
& = & \sum_i^{n_b} \frac{\Sigma_c(i,i)}{\epsilon_i^2}
\end{eqnarray}
Note that this is effectively the sum-of-squares of the inverse CNR values for
each coefficient. An estimate of the overall CNR for the acquisition can
therefore be obtained as $1/\sqrt{\Psi}$. In practice, the CNR values for each
coefficient decrease rapidly, and this overall CNR value will be dominated by
the smallest of these CNR values. Hence, minimising $\Psi$ can also be viewed
as optimising the CNR of noisiest coefficient, without seriously compromising the
CNR of the other coefficients.

To derive $\Sigma_c(i,i)$, we use eqn~\ref{noise_prop} and re-write it as:
\begin{eqnarray}
\Sigma_c(i,i) & = & \sum_j^{n_b} m(i,j) \Sigma_s(j,j) m(i,j) \\
& = & \sigma_s^2 \sum_j \frac{ m(i,j)^2 }{ d_j }
\end{eqnarray}
leading to:
\begin{eqnarray}
\Psi & = & \sigma_s^2 \sum_i^{n_b} \left[ \frac{1}{\epsilon_i^2} \sum_j \frac{ m(i,j)^2 }{ d_j } \right] \\
\end{eqnarray}

To determine the optimal number of directions $\left\{ d_i \right\}$ given a
response matrix $H$ measured over $n_b$ $b$-values $\left\{ b_i \right\}$, we
need to minimise $\Psi$ with respect to $\left\{ d_i \right\}$, with the total
number of DW directions remaining fixed at $\nDW$:
\begin{equation}
\left\{ d_i \right\}_{\textrm{opt}} = \underset{ \left\{ d_i \right\} }{\argmin} \quad \Psi 
\qquad \textrm{such that} \quad \sum_i^{n_b} d_i = \nDW
\end{equation}

Note that without loss of generality, we can set $\nDW = 1$, since $\Psi$ is
directly proportional to $1/\nDW$. The number of directions derived
assuming $\nDW=1$ can then be scaled up to the actual number to be acquired in
practice, with no effect on the relative numbers per shell. 

To identify the optimal $\left\{ d_i \right\}$, we use Lagrangian multipliers to add the constraint into the formulation:
\begin{eqnarray}
\Lagr (\left\{ d_i \right\}, \lambda) & = & \Psi + \lambda \sum_i^{n_b} d_i \\
& = & \sigma_s^2 \sum_i^{n_b} \left[ \frac{1}{\epsilon_i^2} \sum_j \frac{ m(i,j)^2 }{ d_j } \right] + \lambda \sum_i^{n_b} d_i 
\end{eqnarray}

Differentiating with respect to $d_k$:
\begin{eqnarray}
\frac {\partial \Lagr (\left\{ d_i \right\}, \lambda)}{\partial d_k} & = & - \sigma_s^2 \sum_i^{n_b} \frac{1}{\epsilon_i^2} \frac{ m(i,k)^2 }{ d_k^2 } + \lambda  \\
& = & - \frac{\sigma_s^2}{d_k^2} \sum_i^{n_b} \frac{m(i,k)^2}{\epsilon_i^2} + \lambda
\end{eqnarray}

Setting to zero yields:
\begin{eqnarray}
d_k^2 & = & \frac{\sigma_s^2}{\lambda} \sum_i^{n_b} \frac{m(i,k)^2}{\epsilon_i^2} \\
\textrm{and hence} \qquad d_k & \propto & \sqrt {\sum_i^{n_b} \frac{m(i,k)^2}{\epsilon_i^2} }
\end{eqnarray}

It is then a trivial matter to obtain the optimal set of directions using the
above equation, by scaling their sum to the required $\nDW$.


\subsection{Determining the optimal $b$-values}

Now that we have:
\begin{itemize}
\item a way of estimating the intrinsic responses in the data;
\item a method for predicting their corresponding DW signal over any set of $b$-values;
\item a means of identifying the best number of directions to use at each of these $b$-values;
\item and a function by which to assess the relative efficiencies of these schemes ($\Psi$, eqn~\ref{efficiency}),
\end{itemize}
we can determine the best set of $b$-values to use to ensure the most sensitive
detection of these responses. For a given number $n_b$ of $b$-values, the
procedure involves the minimisation of the cost function $\Psi$ with respect to
the $b$-values $\left\{ b_i \right\}_{i\neq 0}$ ($b=0$ is excluded from the
search space since it will always be acquired). This is done using a simple
Nelder-Mead simplex minimisation routine, as implemented in Matlab's
\texttt{fminsearch} routine. 


\subsection{Accounting for T2 decay}

In practice, the use of larger $b$-values will lead to longer echo times, which will
in turn reduce the CNR of the coefficients estimated. Since it is usual
practice in dMRI to vary the $b$-value by altering the DW gradient amplitude,
keeping all other parameters (including sequence timings) identical, this means
that the maximum $b$-value used will influence the CNR of all the estimated
coefficients. This means that the set of optimal $b$-values will be one with a
lower $b_{\textrm{max}}$ than would be expected if $T_2$ effects were
neglected. For a simple pulsed-gradient spin-echo sequence, it is relatively
trivial to derive the minimum echo time (TE) that can be achieved for a given
$b_{\textrm{max}}$, assuming values for (see figure~\ref{PGSE}):
\begin{itemize}
\item $G_{\textrm{max}}$, the maximum gradient amplitude;
\item $t_{\textrm{90}}$, the time between excitation and the onset of the first DW
pulse;
\item $t_{\textrm{180}}$, the time between the 180$^\circ$ refocusing pulse and
the onset of the second DW pulse;
\item and $t_{\textrm{RO}}$, the time between the end
of the last DW gradient pulse and the spin-echo.
\end{itemize}

\begin{figure}[htbp]
\centering
\includegraphics[width=18cm]{PGSE.pdf} 
\caption{The basic pulsed-gradient spin-echo sequence, with the relevant timing parameters labelled.}
\label{PGSE}
\end{figure}

Assuming that the minimum TE is always achieved by ensuring that the second DW 
pulse is applied immediately after the 180$^\circ$ refocusing pulse, and that 
the read-out start immediately after that, the echo time is given by:
\begin{equation}
\label{TE_eqn}
\textrm{TE} = 2 \times ( t_{180} + \delta + t_{\textrm{RO}} )
\end{equation}
The $b$-value is given by:
\begin{equation}
\label{b_eqn}
b_{\textrm{max}} = \gamma^2 G^2 \delta^2 (\Delta - \delta/3)
\end{equation}
Assuming the minimum TE is always achieved by starting the first DW pulse 
immediately after the excitation pulse, $\Delta$ is given by:
\begin{equation}
\Delta = \textrm{TE}/2 + t_{180} - t_{90}
\end{equation}
Substituting into equations~\ref{b_eqn} and \ref{TE_eqn} gives:
\begin{equation}
\delta^3 + \frac{3}{2} \delta^2 (t_{\textrm{RO}} + 2 t_{180} - t_{90}) - \frac{3 b_{\textrm{max}}}{2\gamma^2 G^2} = 0
\end{equation}

This is a cubic equation, the positive root of which provides the value of
$\delta$. The TE can then be obtained using eqn~\ref{TE_eqn}, and the effect
sizes used in the analysis can be scaled down by the factor
$e^{-\frac{\textrm{TE}}{T_2}}$ to account for $T_2$ losses.


\subsection{Data collection}

3 data sets were acquired on neonates with the following parameters: $b = 0,
500, 1000, 2000, 3000, 4000 \, \mms$, 50 DW directions and 1 $b=0$ volume per
shell. Processing was restricted with a mask of the brain. Each dataset was
processed separately to investigate inter-subject variation, and the whole
analysis was also performed on a combined dataset obtained by concatenation of
all 3 individual datasets.


\section{Results}

\subsection{Responses}

\begin{figure}[htbp]
\begin{tabular}{*{7}{c}}
\toprule
& response 1 & response 2 & response 3 & response 4 & response 5 & response 6 \\
\midrule

\multirow{2}{*}{patient 1} & 
\includegraphics[width=2cm]{pat1/weights0000.png} &
\includegraphics[width=2cm]{pat1/weights0001.png} &
\includegraphics[width=2cm]{pat1/weights0002.png} &
\includegraphics[width=2cm]{pat1/weights0003.png} &
\includegraphics[width=2cm]{pat1/weights0004.png} &
\includegraphics[width=2cm]{pat1/weights0005.png} \\
& 
\includegraphics[width=2cm]{pat1/response_1.pdf} &
\includegraphics[width=2cm]{pat1/response_2.pdf} &
\includegraphics[width=2cm]{pat1/response_3.pdf} &
\includegraphics[width=2cm]{pat1/response_4.pdf} &
\includegraphics[width=2cm]{pat1/response_5.pdf} &
\includegraphics[width=2cm]{pat1/response_6.pdf} \\

\multirow{2}{*}{patient 2} & 
\includegraphics[width=2cm]{pat2/weights0000.png} &
\includegraphics[width=2cm]{pat2/weights0001.png} &
\includegraphics[width=2cm]{pat2/weights0002.png} &
\includegraphics[width=2cm]{pat2/weights0003.png} &
\includegraphics[width=2cm]{pat2/weights0004.png} &
\includegraphics[width=2cm]{pat2/weights0005.png} \\
& 
\includegraphics[width=2cm]{pat2/response_1.pdf} &
\includegraphics[width=2cm]{pat2/response_2.pdf} &
\includegraphics[width=2cm]{pat2/response_3.pdf} &
\includegraphics[width=2cm]{pat2/response_4.pdf} &
\includegraphics[width=2cm]{pat2/response_5.pdf} &
\includegraphics[width=2cm]{pat2/response_6.pdf} \\


\multirow{2}{*}{patient 3} & 
\includegraphics[width=2cm]{pat3/weights0000.png} &
\includegraphics[width=2cm]{pat3/weights0001.png} &
\includegraphics[width=2cm]{pat3/weights0002.png} &
\includegraphics[width=2cm]{pat3/weights0003.png} &
\includegraphics[width=2cm]{pat3/weights0004.png} &
\includegraphics[width=2cm]{pat3/weights0005.png} \\
& 
\includegraphics[width=2cm]{pat3/response_1.pdf} &
\includegraphics[width=2cm]{pat3/response_2.pdf} &
\includegraphics[width=2cm]{pat3/response_3.pdf} &
\includegraphics[width=2cm]{pat3/response_4.pdf} &
\includegraphics[width=2cm]{pat3/response_5.pdf} &
\includegraphics[width=2cm]{pat3/response_6.pdf} \\

\midrule

\multirow{1}{*}{combined} & 
\includegraphics[width=2cm]{pat_all/response_1.pdf} &
\includegraphics[width=2cm]{pat_all/response_2.pdf} &
\includegraphics[width=2cm]{pat_all/response_3.pdf} &
\includegraphics[width=2cm]{pat_all/response_4.pdf} &
\includegraphics[width=2cm]{pat_all/response_5.pdf} &
\includegraphics[width=2cm]{pat_all/response_6.pdf} \\

\bottomrule
\end{tabular}
\caption{SVD-derived responses and the corresponding weights for each of the 3 subjects,
 in order of decreasing effect size.}
\end{figure}

\begin{table}[htbp]
\centering
\pgfplotstabletypeset[
  fixed zerofill,
  precision=5,
  create on use/newcol/.style={ create col/set list={patient 1, patient 2, patient 3, combined} },
  columns/newcol/.style={string type, column name={} },
  columns/0/.style={ column name={response 1} },
  columns/1/.style={ column name={response 2} },
  columns/2/.style={ column name={response 3} },
  columns/3/.style={ column name={response 4} },
  columns/4/.style={ column name={response 5} },
  columns/5/.style={ column name={response 6} },
  columns={newcol,0,1,2,3,4,5},
  every head row/.style={before row=\toprule,after row=\midrule},
  every last row/.style={before row=\midrule, after row=\bottomrule},
]{effect_sizes.txt}
\caption{SVD-derived effect sizes for each response, per subject.}
\end{table}






\clearpage
\subsection{Neglecting T2 effects}

\begin{figure}[htbp]
\centering
\begin{tabular}{m{1.5cm}*{3}{m{4cm}}}
\toprule
& \multicolumn{1}{c}{$n_b = n_c = 3$} & \multicolumn{1}{c}{$n_b = n_c = 4$} & \multicolumn{1}{c}{$n_b = n_c = 5$} \\
\midrule
patient 1 & 
\includegraphics[width=4cm]{pat1/results_3b_3coefs.pdf} &
\includegraphics[width=4cm]{pat1/results_4b_4coefs.pdf} &
\includegraphics[width=4cm]{pat1/results_5b_5coefs.pdf} \\

patient 2 & 
\includegraphics[width=4cm]{pat2/results_3b_3coefs.pdf} &
\includegraphics[width=4cm]{pat2/results_4b_4coefs.pdf} &
\includegraphics[width=4cm]{pat2/results_5b_5coefs.pdf} \\

patient 3 & 
\includegraphics[width=4cm]{pat3/results_3b_3coefs.pdf} &
\includegraphics[width=4cm]{pat3/results_4b_4coefs.pdf} &
\includegraphics[width=4cm]{pat3/results_5b_5coefs.pdf} \\

\midrule

combined & 
\includegraphics[width=4cm]{pat_all/results_3b_3coefs.pdf} &
\includegraphics[width=4cm]{pat_all/results_4b_4coefs.pdf} &
\includegraphics[width=4cm]{pat_all/results_5b_5coefs.pdf} \\
\bottomrule
\end{tabular}
\caption{SVD-derived responses for each of the 3 subjects, and with all
subjects combined, with optimal $b$-values denoted by the open circles. 
In each case, the $b$-values were optimised to provide the greatest overall CNR
for the first $n_c = n_b$ coefficients, using $n_b$ distinct $b$-values. The
responses in order of decreasing effect size are shown in blue, green, red,
cyan, magenta, yellow. }
\end{figure}


\clearpage
\subsubsection{2 shells: $n_b = n_c = 3$}

\begin{table}[htbp]
\centering
\pgfplotstabletypeset[
  fixed zerofill,
  precision=0,
  create on use/newcol/.style={ create col/set list={patient 1, patient 2, 
patient 3, combined} },
  columns/newcol/.style={string type, column name={} },
  columns/0/.style={ column name={$b=0$} },
  columns/1/.style={ column name={shell 1} },
  columns/2/.style={ column name={shell 2} },
  columns={newcol,0,1,2},
  every head row/.style={before row=\toprule,after row=\midrule},
  every last row/.style={before row=\midrule, after row=\bottomrule},
]{bvals_opt_3b_3coefs.txt}
\captionbval{2}{3}
\end{table}

\begin{table}[htbp]
\centering
\pgfplotstabletypeset[
  fixed zerofill,
  precision=1,
  create on use/newcol/.style={ create col/set list={patient 1, patient 2, patient 3, combined} },
  columns/newcol/.style={string type, column name={} },
  columns/0/.style={ column name={$b=0$} },
  columns/1/.style={ column name={shell 1} },
  columns/2/.style={ column name={shell 2} },
  columns={newcol,0,1,2},
  every head row/.style={before row=\toprule,after row=\midrule},
  every last row/.style={before row=\midrule, after row=\bottomrule},
]{nDW_opt_3b_3coefs.txt}
\captionnDW{2}{3}
\end{table}


\begin{table}[htbp]
\centering
\pgfplotstabletypeset[
  fixed zerofill,
  precision=2,
  create on use/newcol/.style={ create col/set list={patient 1, patient 2, patient 3, combined} },
  columns/newcol/.style={string type, column name={} },
  columns/0/.style={ column name={coef 1} },
  columns/1/.style={ column name={coef 2} },
  columns/2/.style={ column name={coef 3} },
  columns={newcol,0,1,2},
  every head row/.style={before row=\toprule,after row=\midrule},
  every last row/.style={before row=\midrule, after row=\bottomrule},
]{CNR_opt_3b_3coefs.txt}
\captionCNR{2}{3}
\end{table}






\clearpage
\subsubsection{3 shells: $n_b = n_c = 4$}

\begin{table}[htbp]
\centering
\pgfplotstabletypeset[
  fixed zerofill,
  precision=0,
  create on use/newcol/.style={ create col/set list={patient 1, patient 2, 
patient 3, combined} },
  columns/newcol/.style={string type, column name={} },
  columns/0/.style={ column name={$b=0$} },
  columns/1/.style={ column name={shell 1} },
  columns/2/.style={ column name={shell 2} },
  columns/3/.style={ column name={shell 3} },
  columns={newcol,0,1,2,3},
  every head row/.style={before row=\toprule,after row=\midrule},
  every last row/.style={before row=\midrule, after row=\bottomrule},
]{bvals_opt_4b_4coefs.txt}
\captionbval{3}{4}
\end{table}

\begin{table}[htbp]
\centering
\pgfplotstabletypeset[
  fixed zerofill,
  precision=1,
  create on use/newcol/.style={ create col/set list={patient 1, patient 2, patient 3, combined} },
  columns/newcol/.style={string type, column name={} },
  columns/0/.style={ column name={$b=0$} },
  columns/1/.style={ column name={shell 1} },
  columns/2/.style={ column name={shell 2} },
  columns/3/.style={ column name={shell 3} },
  columns={newcol,0,1,2,3},
  every head row/.style={before row=\toprule,after row=\midrule},
  every last row/.style={before row=\midrule, after row=\bottomrule},
]{nDW_opt_4b_4coefs.txt}
\captionnDW{3}{4}
\end{table}


\begin{table}[htbp]
\centering
\pgfplotstabletypeset[
  fixed zerofill,
  precision=2,
  create on use/newcol/.style={ create col/set list={patient 1, patient 2, patient 3, combined} },
  columns/newcol/.style={string type, column name={} },
  columns/0/.style={ column name={coef 1} },
  columns/1/.style={ column name={coef 2} },
  columns/2/.style={ column name={coef 3} },
  columns/3/.style={ column name={coef 4} },
  columns={newcol,0,1,2,3},
  every head row/.style={before row=\toprule,after row=\midrule},
  every last row/.style={before row=\midrule, after row=\bottomrule},
]{CNR_opt_4b_4coefs.txt}
\captionCNR{3}{4}
\end{table}





\clearpage
\subsubsection{4 shells: $n_b = n_c = 5$}

\begin{table}[htbp]
\centering
\pgfplotstabletypeset[
  fixed zerofill,
  precision=0,
  create on use/newcol/.style={ create col/set list={patient 1, patient 2, 
patient 3, combined} },
  columns/newcol/.style={string type, column name={} },
  columns/0/.style={ column name={$b=0$} },
  columns/1/.style={ column name={shell 1} },
  columns/2/.style={ column name={shell 2} },
  columns/3/.style={ column name={shell 3} },
  columns/4/.style={ column name={shell 4} },
  columns={newcol,0,1,2,3,4},
  every head row/.style={before row=\toprule,after row=\midrule},
  every last row/.style={before row=\midrule, after row=\bottomrule},
]{bvals_opt_5b_5coefs.txt}
\captionbval{4}{5}
\end{table}

\begin{table}[htbp]
\centering
\pgfplotstabletypeset[
  fixed zerofill,
  precision=1,
  create on use/newcol/.style={ create col/set list={patient 1, patient 2, patient 3, combined} },
  columns/newcol/.style={string type, column name={} },
  columns/0/.style={ column name={$b=0$} },
  columns/1/.style={ column name={shell 1} },
  columns/2/.style={ column name={shell 2} },
  columns/3/.style={ column name={shell 3} },
  columns/4/.style={ column name={shell 4} },
  columns={newcol,0,1,2,3,4},
  every head row/.style={before row=\toprule,after row=\midrule},
  every last row/.style={before row=\midrule, after row=\bottomrule},
]{nDW_opt_5b_5coefs.txt}
\captionnDW{4}{5}
\end{table}


\begin{table}[htbp]
\centering
\pgfplotstabletypeset[
  fixed zerofill,
  precision=2,
  create on use/newcol/.style={ create col/set list={patient 1, patient 2, patient 3, combined} },
  columns/newcol/.style={string type, column name={} },
  columns/0/.style={ column name={coef 1} },
  columns/1/.style={ column name={coef 2} },
  columns/2/.style={ column name={coef 3} },
  columns/3/.style={ column name={coef 4} },
  columns/4/.style={ column name={coef 5} },
  columns={newcol,0,1,2,3,4},
  every head row/.style={before row=\toprule,after row=\midrule},
  every last row/.style={before row=\midrule, after row=\bottomrule},
]{CNR_opt_5b_5coefs.txt}
\captionCNR{4}{5}
\end{table}








\clearpage
\subsection{Assuming $T_2 = 80\,ms$}

\begin{figure}[htbp]
\centering
\begin{tabular}{m{1.5cm}*{3}{m{4cm}}}
\toprule
& \multicolumn{1}{c}{$n_b = n_c = 3$} & \multicolumn{1}{c}{$n_b = n_c = 4$} & \multicolumn{1}{c}{$n_b = n_c = 5$} \\
\midrule
patient 1 & 
\includegraphics[width=4cm]{pat1/results_3b_3coefs_T2_80ms.pdf} &
\includegraphics[width=4cm]{pat1/results_4b_4coefs_T2_80ms.pdf} &
\includegraphics[width=4cm]{pat1/results_5b_5coefs_T2_80ms.pdf} \\

patient 2 & 
\includegraphics[width=4cm]{pat2/results_3b_3coefs_T2_80ms.pdf} &
\includegraphics[width=4cm]{pat2/results_4b_4coefs_T2_80ms.pdf} &
\includegraphics[width=4cm]{pat2/results_5b_5coefs_T2_80ms.pdf} \\

patient 3 & 
\includegraphics[width=4cm]{pat3/results_3b_3coefs_T2_80ms.pdf} &
\includegraphics[width=4cm]{pat3/results_4b_4coefs_T2_80ms.pdf} &
\includegraphics[width=4cm]{pat3/results_5b_5coefs_T2_80ms.pdf} \\

\midrule

combined & 
\includegraphics[width=4cm]{pat_all/results_3b_3coefs_T2_80ms.pdf} &
\includegraphics[width=4cm]{pat_all/results_4b_4coefs_T2_80ms.pdf} &
\includegraphics[width=4cm]{pat_all/results_5b_5coefs_T2_80ms.pdf} \\
\bottomrule
\end{tabular}
\caption{SVD-derived responses for each of the 3 subjects, and with all
subjects combined, with optimal $b$-values denoted by the open circles. 
\textbf{In this case b-values were optimised taking into account the minimum TE
achievable for the corresponding $b_\textrm{max}$, assuming a TE = 80ms}.
In each case, the $b$-values were optimised to provide the greatest overall CNR
for the first $n_c = n_b$ coefficients, using $n_b$ distinct $b$-values. The
responses in order of decreasing effect size are shown in blue, green, red,
cyan, magenta, yellow. }
\end{figure}


\clearpage
\subsubsection{2 shells: $n_b = n_c = 3$}

\begin{table}[htbp]
\centering
\pgfplotstabletypeset[
  fixed zerofill,
  precision=0,
  create on use/newcol/.style={ create col/set list={patient 1, patient 2, 
patient 3, combined} },
  columns/newcol/.style={string type, column name={} },
  columns/0/.style={ column name={$b=0$} },
  columns/1/.style={ column name={shell 1} },
  columns/2/.style={ column name={shell 2} },
  columns={newcol,0,1,2},
  every head row/.style={before row=\toprule,after row=\midrule},
  every last row/.style={before row=\midrule, after row=\bottomrule},
]{bvals_opt_3b_3coefs_T2_80ms.txt}
\captionbvalTE{2}{3}{80}
\end{table}

\begin{table}[htbp]
\centering
\pgfplotstabletypeset[
  fixed zerofill,
  precision=1,
  create on use/newcol/.style={ create col/set list={patient 1, patient 2, patient 3, combined} },
  columns/newcol/.style={string type, column name={} },
  columns/0/.style={ column name={$b=0$} },
  columns/1/.style={ column name={shell 1} },
  columns/2/.style={ column name={shell 2} },
  columns={newcol,0,1,2},
  every head row/.style={before row=\toprule,after row=\midrule},
  every last row/.style={before row=\midrule, after row=\bottomrule},
]{nDW_opt_3b_3coefs_T2_80ms.txt}
\captionnDWTE{2}{3}{80}
\end{table}


\begin{table}[htbp]
\centering
\pgfplotstabletypeset[
  fixed zerofill,
  precision=2,
  create on use/newcol/.style={ create col/set list={patient 1, patient 2, patient 3, combined} },
  columns/newcol/.style={string type, column name={} },
  columns/0/.style={ column name={coef 1} },
  columns/1/.style={ column name={coef 2} },
  columns/2/.style={ column name={coef 3} },
  columns={newcol,0,1,2},
  every head row/.style={before row=\toprule,after row=\midrule},
  every last row/.style={before row=\midrule, after row=\bottomrule},
]{CNR_opt_3b_3coefs_T2_80ms.txt}
\captionCNRTE{2}{3}{80}
\end{table}






\clearpage
\subsubsection{3 shells: $n_b = n_c = 4$}

\begin{table}[htbp]
\centering
\pgfplotstabletypeset[
  fixed zerofill,
  precision=0,
  create on use/newcol/.style={ create col/set list={patient 1, patient 2, 
patient 3, combined} },
  columns/newcol/.style={string type, column name={} },
  columns/0/.style={ column name={$b=0$} },
  columns/1/.style={ column name={shell 1} },
  columns/2/.style={ column name={shell 2} },
  columns/3/.style={ column name={shell 3} },
  columns={newcol,0,1,2,3},
  every head row/.style={before row=\toprule,after row=\midrule},
  every last row/.style={before row=\midrule, after row=\bottomrule},
]{bvals_opt_4b_4coefs_T2_80ms.txt}
\captionbvalTE{3}{4}{80}
\end{table}

\begin{table}[htbp]
\centering
\pgfplotstabletypeset[
  fixed zerofill,
  precision=1,
  create on use/newcol/.style={ create col/set list={patient 1, patient 2, patient 3, combined} },
  columns/newcol/.style={string type, column name={} },
  columns/0/.style={ column name={$b=0$} },
  columns/1/.style={ column name={shell 1} },
  columns/2/.style={ column name={shell 2} },
  columns/3/.style={ column name={shell 3} },
  columns={newcol,0,1,2,3},
  every head row/.style={before row=\toprule,after row=\midrule},
  every last row/.style={before row=\midrule, after row=\bottomrule},
]{nDW_opt_4b_4coefs_T2_80ms.txt}
\captionnDWTE{3}{4}{80}
\end{table}


\begin{table}[htbp]
\centering
\pgfplotstabletypeset[
  fixed zerofill,
  precision=2,
  create on use/newcol/.style={ create col/set list={patient 1, patient 2, patient 3, combined} },
  columns/newcol/.style={string type, column name={} },
  columns/0/.style={ column name={coef 1} },
  columns/1/.style={ column name={coef 2} },
  columns/2/.style={ column name={coef 3} },
  columns/3/.style={ column name={coef 4} },
  columns={newcol,0,1,2,3},
  every head row/.style={before row=\toprule,after row=\midrule},
  every last row/.style={before row=\midrule, after row=\bottomrule},
]{CNR_opt_4b_4coefs_T2_80ms.txt}
\captionCNRTE{3}{4}{80}
\end{table}





\clearpage
\subsubsection{4 shells: $n_b = n_c = 5$}

\begin{table}[htbp]
\centering
\pgfplotstabletypeset[
  fixed zerofill,
  precision=0,
  create on use/newcol/.style={ create col/set list={patient 1, patient 2, 
patient 3, combined} },
  columns/newcol/.style={string type, column name={} },
  columns/0/.style={ column name={$b=0$} },
  columns/1/.style={ column name={shell 1} },
  columns/2/.style={ column name={shell 2} },
  columns/3/.style={ column name={shell 3} },
  columns/4/.style={ column name={shell 4} },
  columns={newcol,0,1,2,3,4},
  every head row/.style={before row=\toprule,after row=\midrule},
  every last row/.style={before row=\midrule, after row=\bottomrule},
]{bvals_opt_5b_5coefs_T2_80ms.txt}
\captionbvalTE{4}{5}{80}
\end{table}

\begin{table}[htbp]
\centering
\pgfplotstabletypeset[
  fixed zerofill,
  precision=1,
  create on use/newcol/.style={ create col/set list={patient 1, patient 2, patient 3, combined} },
  columns/newcol/.style={string type, column name={} },
  columns/0/.style={ column name={$b=0$} },
  columns/1/.style={ column name={shell 1} },
  columns/2/.style={ column name={shell 2} },
  columns/3/.style={ column name={shell 3} },
  columns/4/.style={ column name={shell 4} },
  columns={newcol,0,1,2,3,4},
  every head row/.style={before row=\toprule,after row=\midrule},
  every last row/.style={before row=\midrule, after row=\bottomrule},
]{nDW_opt_5b_5coefs_T2_80ms.txt}
\captionnDWTE{4}{5}{80}
\end{table}


\begin{table}[htbp]
\centering
\pgfplotstabletypeset[
  fixed zerofill,
  precision=2,
  create on use/newcol/.style={ create col/set list={patient 1, patient 2, patient 3, combined} },
  columns/newcol/.style={string type, column name={} },
  columns/0/.style={ column name={coef 1} },
  columns/1/.style={ column name={coef 2} },
  columns/2/.style={ column name={coef 3} },
  columns/3/.style={ column name={coef 4} },
  columns/4/.style={ column name={coef 5} },
  columns={newcol,0,1,2,3,4},
  every head row/.style={before row=\toprule,after row=\midrule},
  every last row/.style={before row=\midrule, after row=\bottomrule},
]{CNR_opt_5b_5coefs_T2_80ms.txt}
\captionCNRTE{4}{5}{80}
\end{table}





\end{document}



