\documentclass{article}

\usepackage[pdftex]{color,graphicx}
%\usepackage{lmodern}
%\usepackage[T1]{fontenc}
%\usepackage{textcomp}
\usepackage[margin=2cm]{geometry}
%\usepackage[space]{grffile}
%\usepackage{adjustbox}
\usepackage{array}
\usepackage{multirow}
\usepackage{booktabs} % For \toprule, \midrule and \bottomrule
\usepackage{siunitx} % Formats the units and values
\usepackage{pgfplotstable} % Generates table from .csv

% Setup siunitx:
\sisetup{
  round-mode          = places, 
  round-precision     = 3, 
}

\newcommand{\nDW}{n_{\textrm{DW}}}
\newcommand{\nvox}{n_{\textrm{vox}}}
\newcommand{\mms}{\textrm{mm}^2/\textrm{s}}

\author{J-D.~Tournier}
\title{optimisation of dHCP diffusion acquisition protocol}

\begin{document}
\maketitle

\section{Overview}

Basic idea: the DW signal within each voxel is a linear combination of tissue-specific
signatures. These tissue signatures may have orientational features (notably
for white matter), but what is critical for the determination of the `optimal'
multi-shell scheme is that the DW signal averaged over all orientations (i.e.
the 'DC' term) have distinct features as a function of $b$-value. 

The linear combination approach is generally assumed valid in the limit of
negligible exchange between tissue types - this is used in many complex models
of DW signal in white matter. It's completely valid when the different
components correspond to physically distant regions of tissue, for example when
two tracts pass close to each other (leading to partial volume averaging),
since the water molecules will not be able to diffuse between compartments
during the applied diffusion time. It is also probably fine even when different
compartment co-exist given that the pre-exchange lifetime of water molecules in
cells is typically estimated at between 0.5 to 1s, which is long compared to
the $\sim$50ms diffusion times used. 

The focus on the isotropic part of the signal, ignoring orientational features,
is justified as follows. First, in the absence of other information, any
orientation feature in the `intrinsic' DW signal cannot be disentangled from
the orientation distribution function (ODF) for that tissue in that voxel - in
other words, orientation features may change simply due to a rotation of the
tissue, or the presence of multiple orientations of it within the same voxel
(cf. crossing fibres). On the other hand, the DC term of that signal is
rotationally invariant, and directly proportional to the amount of
corresponding tissue. Second, while focusing on the isotropic part may discard
usable information (in terms of teasing apart the various compartments), this
implies that the contrast-to-noise measures estimated by this procedure are 
conservative -- it may be possible to improve on the values given, but they
nonetheless constitute a sane lower bound. 

\subsection{General approach}

Based on the above, the idea is to factorise the DW signal into 3 parts: 
\begin{itemize}
\item a set of intrinsic tissue-specific {\em responses};
\item their corresponding {\em effect sizes} (i.e. how much they contribute to the signal for a `unit' response);
\item a vector of voxel-wise {\em weights} giving the amount of each response per voxel.
\end{itemize}

Given such a factorisation, and a set of $b$-values to use, it is possible to
identify the optimal number of directions $\nDW$ per $b$-value (as a fraction of the
total number acquired), such that the contrast-to-noise ratio (CNR) of the
estimated weights is maximised. 

Identifying the optimal set of $b$-values then becomes a matter of optimising
for maximum CNR (given optimised $\nDW$ as above).

\subsection{Signal factorisation}

There are a variety of ways to achieve this, with the most intuitive being the
non-negative matrix factorisation approach; unfortunately this turns out to
have stability issues. A more tried-and-tested matrix decomposition is the
singular value decomposition (SVD), which actually has some nice properties --
not least of which is the fact it does directly break down the signal as above. 

Given a $\nvox \times n_b$ matrix of data values $D$, we use the
slim SVD to provide matrices $W$, $E$ and $H$ such that $D = W E H^T$, where:
\begin{itemize}
\item $W$ is the $\nvox \times n_c$ matrix of weights for each coefficient, per voxel;
\item $E$ is the $n_c \times n_c$ diagonal matrix of effect sizes for each coefficient;
\item $H$ is the $n_b \times n_c$ orthogonal matrix of response function,
giving the expected signal for a unit response at each $b$-value.
\end{itemize}
In the above, $n_b$ corresponds to the number of $b$-values in the acquisition
(including the $b=0$), $n_c$ to the number of response functions in the data,
and $\nvox$ to the number of voxels within the image mask. 

To achieve the above, the matrices $D = USV^T$ of the slim SVD needs to be
modified, since the columns of $U$ are normalised to unity, whereas we need the
elements of $W$ to have unit root-mean-square value -- this is necessary to
ensure that the effect sizes are meaningful. To achieve this, the SVD
output is modified as follows: 
\begin{eqnarray}
W & = & U \times \sqrt{\nvox} \\
E & = & S / \sqrt{\nvox} \\
H & = & V
\end{eqnarray}


\subsubsection{Interpolation over the $b$-value domain}
In order to enable a more fine-grained optimisation of the $b$-values,
and allow the estimation of contrast to noise at $b$-values other than those
specified, a suitable interpolation approach is required. There are many ways
to achieve this; here we chose to use the following: 
\begin{enumerate}
\item the data matrix $D$ is interpolated along the $b$-value dimension using
the Piecewise Cubic Hermite Interpolating Polynomial (PCHIP) interpolation
routine provided in Matlab, to a final resolution $\Delta_b = 100 \mms$, to
produce the $\nvox \times n_{b'}$ matrix $D'$, with $n_{b'} = 1 + b_{\textrm{max}} / \Delta_b$ 
($n_{b'} = 41$ in this case). The PCHIP algorithm was chosen based on its
guaranteed monotonicity when applied to monotonic data (as is the case here).
\item the slim SVD is performed on the interpolated matrix $D'$, but only the
sub-matrices corresponding to the first $n_b$ singular values are retained. The
matrix $H$ is therefore now $n_c \times n_{b'}$, with the size of the other
matrices remaining unaffected.
\item responses for a given set of arbitrary $b$-values can then be derived
using a further PCHIP interpolation step on the response matrix $H$.
\end{enumerate} 

\subsection{Contrast to noise ratio estimation}

\subsubsection{Noise Propagation}
Given a problem $c = M s$, the variance-covariance matrix $\Sigma_c$ on the coefficients $c$ is given by:
\begin{equation}
\label{noise_prop}
\Sigma_c = M \Sigma_s M^T 
\end{equation}
where $\Sigma_s$ is the variance-covariance matrix on the measured data $s$.  
Assuming the data are independent and identically distributed (i.i.d.),
$\Sigma_s = \sigma_s^2 I$, where $I$ is the identity matrix, and $\sigma_s$ is
the standard deviation of the imaging noise. In our problem, the matrix $M$
corresponds to $(HE)^\dagger$, the Moore-Penrose Pseudo-inverse of $EH$, given by:
\begin{eqnarray}
(HE)^\dagger & = & \left( (HE)^T(HE) \right)^{-1}(HE)^T \\
& = & \left(E^T H^T H E \right)^{-1} E^T H^T\\
& = & (E H^T H E)^{-1} E H^T \qquad \textrm{if $E$ is diagonal} \\
& = & E^{-1} (H^TH)^{-1} E^{-1} E H^T \qquad \textrm{if $E$ is square invertible} \\
& = & E^{-1} (H^T H)^{-1} H^T \\
& = & E^{-1} H^\dagger
\end{eqnarray}

When multiple measurements of the same data point are available, the results is
essentially to reduce the variance in the measurement by the corresponding
factor -- i.e. $\Sigma_s(i,i) = \sigma_s^2 / n_i$, where $n_i$ is the number of
measurements performed for data point $i$. In the context of the present
analysis, repeated measurements at the same $b$-value will typically be taken
along different DW directions, and so will not correspond. However, provided
the total number of measurements taken at each $b$-value exceeds the minimum
necessary for aliasing-free reconstruction, the variance of the DC term
estimated for that $b$-value is in fact characterised by the same relationship
(Tournier et al., NMR Biomed. 2013). In other words, assuming the sampling of
the orientation domain is uniform and sufficiently dense, the variance of the
mean DW signal at that $b$-value is also inversely proportional to the
number of measurements performed at that $b$-value. 

For the present study, a spherical harmonic order $l = 6$ was shown to be
sufficient to fully characterise the DW signal at high $b$-values, implying
that a set of 28 directions was sufficient to ensure sufficiently dense
sampling. This number is reduced to order $l=4$ (15 DW directions) for lower
$b$-values, and eventually to $l=2$ (6 measurements) and $l=0$ (1 measurement)
at $b=0$. Hence, the above argument holds provided the number of measurement at
each $b$-value is greater than those numbers. 

\subsubsection{Determining the optimum number of DW directions}


Eqn~\ref{noise_prop} can be re-written as:
\begin{eqnarray}
\Sigma_c(i,i) & = & \sum_j m(i,j) \Sigma_s(j,j) m(i,j) \\
& = & \sum_j \frac{ m(i,j)^2 \sigma^2_s(j) }{ n_j }
\end{eqnarray}




\section{Results}

\subsection{Responses}

\begin{figure}[htbp]
\begin{tabular}{*{7}{c}}
\toprule
& response 1 & response 2 & response 3 & response 4 & response 5 & response 6 \\
\midrule

\multirow{2}{*}{patient 1} & 
\includegraphics[width=2cm]{pat1/weights0000.png} &
\includegraphics[width=2cm]{pat1/weights0001.png} &
\includegraphics[width=2cm]{pat1/weights0002.png} &
\includegraphics[width=2cm]{pat1/weights0003.png} &
\includegraphics[width=2cm]{pat1/weights0004.png} &
\includegraphics[width=2cm]{pat1/weights0005.png} \\
& 
\includegraphics[width=2cm]{pat1/response_1.pdf} &
\includegraphics[width=2cm]{pat1/response_2.pdf} &
\includegraphics[width=2cm]{pat1/response_3.pdf} &
\includegraphics[width=2cm]{pat1/response_4.pdf} &
\includegraphics[width=2cm]{pat1/response_5.pdf} &
\includegraphics[width=2cm]{pat1/response_6.pdf} \\


\multirow{2}{*}{patient 2} & 
\includegraphics[width=2cm]{pat2/weights0000.png} &
\includegraphics[width=2cm]{pat2/weights0001.png} &
\includegraphics[width=2cm]{pat2/weights0002.png} &
\includegraphics[width=2cm]{pat2/weights0003.png} &
\includegraphics[width=2cm]{pat2/weights0004.png} &
\includegraphics[width=2cm]{pat2/weights0005.png} \\
& 
\includegraphics[width=2cm]{pat2/response_1.pdf} &
\includegraphics[width=2cm]{pat2/response_2.pdf} &
\includegraphics[width=2cm]{pat2/response_3.pdf} &
\includegraphics[width=2cm]{pat2/response_4.pdf} &
\includegraphics[width=2cm]{pat2/response_5.pdf} &
\includegraphics[width=2cm]{pat2/response_6.pdf} \\


\multirow{2}{*}{patient 3} & 
\includegraphics[width=2cm]{pat3/weights0000.png} &
\includegraphics[width=2cm]{pat3/weights0001.png} &
\includegraphics[width=2cm]{pat3/weights0002.png} &
\includegraphics[width=2cm]{pat3/weights0003.png} &
\includegraphics[width=2cm]{pat3/weights0004.png} &
\includegraphics[width=2cm]{pat3/weights0005.png} \\
& 
\includegraphics[width=2cm]{pat3/response_1.pdf} &
\includegraphics[width=2cm]{pat3/response_2.pdf} &
\includegraphics[width=2cm]{pat3/response_3.pdf} &
\includegraphics[width=2cm]{pat3/response_4.pdf} &
\includegraphics[width=2cm]{pat3/response_5.pdf} &
\includegraphics[width=2cm]{pat3/response_6.pdf} \\

\midrule

\multirow{1}{*}{combined} & 
\includegraphics[width=2cm]{pat_all/response_1.pdf} &
\includegraphics[width=2cm]{pat_all/response_2.pdf} &
\includegraphics[width=2cm]{pat_all/response_3.pdf} &
\includegraphics[width=2cm]{pat_all/response_4.pdf} &
\includegraphics[width=2cm]{pat_all/response_5.pdf} &
\includegraphics[width=2cm]{pat_all/response_6.pdf} \\

\bottomrule
\end{tabular}
\caption{SVD-derived responses and the corresponding weights for each of the 3 subjects,
 in order of decreasing effect size.}
\end{figure}

\begin{table}[htbp]
\centering
\pgfplotstabletypeset[
  fixed zerofill,
  precision=5,
  create on use/newcol/.style={ create col/set list={patient 1, patient 2, patient 3} },
  columns/newcol/.style={string type, column name={} },
  columns/0/.style={ column name={response 1} },
  columns/1/.style={ column name={response 2} },
  columns/2/.style={ column name={response 3} },
  columns/3/.style={ column name={response 4} },
  columns/4/.style={ column name={response 5} },
  columns/5/.style={ column name={response 6} },
  columns={newcol,0,1,2,3,4,5},
  every head row/.style={before row=\toprule,after row=\midrule},
  every last row/.style={after row=\bottomrule},
]{effect_sizes.txt}
\caption{SVD-derived effect sizes for each response, per subject.}
\end{table}

\newpage
\section*{Neglecting T2 effects}

\begin{figure}[htbp]
\begin{tabular}{m{1cm}*{3}{m{5cm}} }
\hline
& \multicolumn{1}{c}{patient 1} & \multicolumn{1}{c}{patient 2} & \multicolumn{1}{c}{patient 3} \\
\hline
$n_b = 3$ \newline $n_c = 3$ &
\includegraphics[width=5cm]{pat1/results_3b_3coefs.pdf} &
\includegraphics[width=5cm]{pat2/results_3b_3coefs.pdf} &
\includegraphics[width=5cm]{pat3/results_3b_3coefs.pdf} \\
$n_b = 4$ \newline $n_c = 4$  &
\includegraphics[width=5cm]{pat1/results_4b_4coefs.pdf} &
\includegraphics[width=5cm]{pat2/results_4b_4coefs.pdf} &
\includegraphics[width=5cm]{pat3/results_4b_4coefs.pdf} \\
$n_b = 5$ \newline $n_c = 5$  &
\includegraphics[width=5cm]{pat1/results_5b_5coefs.pdf} &
\includegraphics[width=5cm]{pat2/results_5b_5coefs.pdf} &
\includegraphics[width=5cm]{pat3/results_5b_5coefs.pdf} \\
\hline
\end{tabular}
\caption{SVD-derived responses for each of the 3 subjects, with optimal b-values denoted by the open circles. 
In each case, the b-values were optimised to provide the greatest overall CNR for the first $n_c$ coefficients, 
using $n_b$ distinct b-values. The responses in order of decreasing effect size are shown in blue, green, red, cyan, magenta, yellow. }
\label{results_noTE}
\end{figure}


\begin{table}[htbp]
\centering
\pgfplotstabletypeset[
  fixed zerofill,
  precision=5,
  create on use/newcol/.style={ create col/set list={patient 1, patient 2, 
patient 3, combined} },
  columns/newcol/.style={string type, column name={} },
  columns/1/.style={ column name={shell 1 (s/mm$^2$)} },
  columns/2/.style={ column name={shell 2 (s/mm$^2$)} },
  columns={newcol,1,2},
  every head row/.style={before row=\toprule,after row=\midrule},
  every last row/.style={before row=\midrule, after row=\bottomrule},
]{bvals_opt_3b_3coefs.txt}
\caption{$b$-values for a 2 shell acquisition (excluding $b=0$), targeting 3 
coefficients, neglecting T2 effects, per subject.}
\end{table}

\begin{table}[htbp]
\centering
\pgfplotstabletypeset[
  fixed zerofill,
  precision=5,
  create on use/newcol/.style={ create col/set list={patient 1, patient 2, patient 3} },
  columns/newcol/.style={string type, column name={} },
  columns/1/.style={ column name={shell 1} },
  columns/2/.style={ column name={shell 2} },
  columns={newcol,1,2},
  every head row/.style={before row=\toprule,after row=\midrule},
  every last row/.style={after row=\bottomrule},
]{nDW_opt_3b_3coefs.txt}
\caption{Number of directions per shell ($n_{DW}$) for a 2 shell acquisition, 
targeting 3 coefficients, neglecting T2 effects, per subject.
This is calculated assuming a total of 400 DW directions, and SNR in the $b=0$ 
image = 15.}
\end{table}

\begin{table}[htbp]
\centering
\pgfplotstabletypeset[
  fixed zerofill,
  precision=5,
  create on use/newcol/.style={ create col/set list={patient 1, patient 2, patient 3} },
  columns/newcol/.style={string type, column name={} },
  columns/0/.style={ column name={coef 1} },
  columns/1/.style={ column name={coef 2} },
  columns/2/.style={ column name={coef 3} },
  columns={newcol,0,1,2},
  every head row/.style={before row=\toprule,after row=\midrule},
  every last row/.style={after row=\bottomrule},
]{CNR_opt_3b_3coefs.txt}
\caption{Optimal contrast to noise ratio per coefficient, for a set of 3 shell, 
estimating 3 coefficients, neglecting T2 effects, per subject.
This assumes a total of 400 DW directions, and SNR in the $b=0$ image = 15.}
\end{table}

\begin{table}[htbp]
\centering
\pgfplotstabletypeset[
  fixed zerofill,
  precision=5,
  create on use/newcol/.style={ create col/set list={patient 1, patient 2, patient 3} },
  columns/newcol/.style={string type, column name={} },
  columns/0/.style={ column name={shell 1} },
  columns/1/.style={ column name={shell 2} },
  columns/2/.style={ column name={shell 3} },
  columns/3/.style={ column name={shell 4} },
  columns={newcol,0,1,2,3},
  every head row/.style={before row=\toprule,after row=\midrule},
  every last row/.style={after row=\bottomrule},
]{bvals_opt_4b_4coefs.txt}
\caption{Optimal b-values for a set of 4 shell, estimating 4 coefficients, 
neglecting T2 effects, per subject.}
\end{table}


\begin{table}[htbp]
\centering
\pgfplotstabletypeset[
  fixed zerofill,
  precision=5,
  create on use/newcol/.style={ create col/set list={patient 1, patient 2, patient 3} },
  columns/newcol/.style={string type, column name={} },
  columns/0/.style={ column name={shell 1} },
  columns/1/.style={ column name={shell 2} },
  columns/2/.style={ column name={shell 3} },
  columns/3/.style={ column name={shell 4} },
  columns={newcol,0,1,2,3},
  every head row/.style={before row=\toprule,after row=\midrule},
  every last row/.style={after row=\bottomrule},
]{nDW_opt_4b_4coefs.txt}
\caption{Optimal number of directions per shell for a set of 4 shell, 
estimating 4 coefficients, neglecting T2 effects, per subject.
This assumes a total of 400 DW directions, and SNR in the $b=0$ image = 15.}
\end{table}

\begin{table}[htbp]
\centering
\pgfplotstabletypeset[
  fixed zerofill,
  precision=5,
  create on use/newcol/.style={ create col/set list={patient 1, patient 2, patient 3} },
  columns/newcol/.style={string type, column name={} },
  columns/0/.style={ column name={coef 1} },
  columns/1/.style={ column name={coef 2} },
  columns/2/.style={ column name={coef 3} },
  columns/3/.style={ column name={coef 4} },
  columns={newcol,0,1,2,3},
  every head row/.style={before row=\toprule,after row=\midrule},
  every last row/.style={after row=\bottomrule},
]{CNR_opt_4b_4coefs.txt}
\caption{Optimal contrast to noise ratio per coefficient, for a set of 4 shell, 
estimating 4 coefficients, neglecting T2 effects, per subject.
This assumes a total of 400 DW directions, and SNR in the $b=0$ image = 15.}
\end{table}

\begin{table}[htbp]
\centering
\pgfplotstabletypeset[
  fixed zerofill,
  precision=5,
  create on use/newcol/.style={ create col/set list={patient 1, patient 2, patient 3} },
  columns/newcol/.style={string type, column name={} },
  columns/0/.style={ column name={shell 1} },
  columns/1/.style={ column name={shell 2} },
  columns/2/.style={ column name={shell 3} },
  columns/3/.style={ column name={shell 4} },
  columns/4/.style={ column name={shell 5} },
  columns={newcol,0,1,2,3,4},
  every head row/.style={before row=\toprule,after row=\midrule},
  every last row/.style={after row=\bottomrule},
]{bvals_opt_5b_5coefs.txt}
\caption{Optimal b-values for a set of 5 shells, estimating 5 coefficients, 
neglecting T2 effects, per subject.}
\end{table}


\begin{table}[htbp]
\centering
\pgfplotstabletypeset[
  fixed zerofill,
  precision=5,
  create on use/newcol/.style={ create col/set list={patient 1, patient 2, patient 3} },
  columns/newcol/.style={string type, column name={} },
  columns/0/.style={ column name={shell 1} },
  columns/1/.style={ column name={shell 2} },
  columns/2/.style={ column name={shell 3} },
  columns/3/.style={ column name={shell 4} },
  columns/4/.style={ column name={shell 5} },
  columns={newcol,0,1,2,3,4},
  every head row/.style={before row=\toprule,after row=\midrule},
  every last row/.style={after row=\bottomrule},
]{nDW_opt_5b_5coefs.txt}
\caption{Optimal number of directions per shell for a set of 5 shells, 
estimating 5 coefficients, neglecting T2 effects, per subject.
This assumes a total of 400 DW directions, and SNR in the $b=0$ image = 15.}
\end{table}

\begin{table}[htbp]
\centering
\pgfplotstabletypeset[
  fixed zerofill,
  precision=5,
  create on use/newcol/.style={ create col/set list={patient 1, patient 2, patient 3} },
  columns/newcol/.style={string type, column name={} },
  columns/0/.style={ column name={coef 1} },
  columns/1/.style={ column name={coef 2} },
  columns/2/.style={ column name={coef 3} },
  columns/3/.style={ column name={coef 4} },
  columns/4/.style={ column name={coef 5} },
  columns={newcol,0,1,2,3,4},
  every head row/.style={before row=\toprule,after row=\midrule},
  every last row/.style={after row=\bottomrule},
]{CNR_opt_5b_5coefs.txt}
\caption{Optimal contrast to noise ratio per coefficient, for a set of 5 shell, 
estimating 5 coefficients, neglecting T2 effects, per subject.
This assumes a total of 400 DW directions, and SNR in the $b=0$ image = 15.}
\end{table}

\newpage
\section*{Assuming T2=80ms}

\begin{figure}[htbp]
\begin{tabular}{m{1cm}*{3}{m{5cm}} }
\hline
& \multicolumn{1}{c}{patient 1} & \multicolumn{1}{c}{patient 2} & \multicolumn{1}{c}{patient 3} \\
\hline
$n_b = 3$ \newline $n_c = 3$ &
\includegraphics[width=5cm]{pat1/results_3b_3coefs_T2_80ms.pdf} &
\includegraphics[width=5cm]{pat2/results_3b_3coefs_T2_80ms.pdf} &
\includegraphics[width=5cm]{pat3/results_3b_3coefs_T2_80ms.pdf} \\
$n_b = 4$ \newline $n_c = 4$  &
\includegraphics[width=5cm]{pat1/results_4b_4coefs_T2_80ms.pdf} &
\includegraphics[width=5cm]{pat2/results_4b_4coefs_T2_80ms.pdf} &
\includegraphics[width=5cm]{pat3/results_4b_4coefs_T2_80ms.pdf} \\
$n_b = 5$ \newline $n_c = 5$  &
\includegraphics[width=5cm]{pat1/results_5b_5coefs_T2_80ms.pdf} &
\includegraphics[width=5cm]{pat2/results_5b_5coefs_T2_80ms.pdf} &
\includegraphics[width=5cm]{pat3/results_5b_5coefs_T2_80ms.pdf} \\
\hline
\end{tabular}
\caption{SVD-derived responses for each of the 3 subjects, with optimal b-values denoted by the open circles. 
\textbf{In this case b-values were optimised taking into account the minimum TE achievable for the corresponding $b_\textrm{max}$, assuming a TE = 80ms}.
In each case, the b-values were optimised to provide the greatest overall CNR for the first $n_c$ coefficients, 
using $n_b$ distinct b-values. The responses in order of decreasing effect size are shown in blue, green, red, cyan, magenta, yellow. }
\end{figure}


\begin{table}[htbp]
\centering
\pgfplotstabletypeset[
  fixed zerofill,
  precision=5,
  create on use/newcol/.style={ create col/set list={patient 1, patient 2, patient 3} },
  columns/newcol/.style={string type, column name={} },
  columns/0/.style={ column name={shell 1} },
  columns/1/.style={ column name={shell 2} },
  columns/2/.style={ column name={shell 3} },
  columns={newcol,0,1,2},
  every head row/.style={before row=\toprule,after row=\midrule},
  every last row/.style={after row=\bottomrule},
]{bvals_opt_3b_3coefs_T2_80ms.txt}
\caption{Optimal b-values for a set of 3 shell, estimating 3 coefficients, 
assuming T2=80ms, per subject.}
\end{table}


\begin{table}[htbp]
\centering
\pgfplotstabletypeset[
  fixed zerofill,
  precision=5,
  create on use/newcol/.style={ create col/set list={patient 1, patient 2, patient 3} },
  columns/newcol/.style={string type, column name={} },
  columns/0/.style={ column name={shell 1} },
  columns/1/.style={ column name={shell 2} },
  columns/2/.style={ column name={shell 3} },
  columns={newcol,0,1,2},
  every head row/.style={before row=\toprule,after row=\midrule},
  every last row/.style={after row=\bottomrule},
]{nDW_opt_3b_3coefs_T2_80ms.txt}
\caption{Optimal number of directions per shell for a set of 3 shell, 
estimating 3 coefficients, assuming T2=80ms, per subject.
This assumes a total of 400 DW directions, and SNR in the $b=0$ image = 15.}
\end{table}

\begin{table}[htbp]
\centering
\pgfplotstabletypeset[
  fixed zerofill,
  precision=5,
  create on use/newcol/.style={ create col/set list={patient 1, patient 2, patient 3} },
  columns/newcol/.style={string type, column name={} },
  columns/0/.style={ column name={coef 1} },
  columns/1/.style={ column name={coef 2} },
  columns/2/.style={ column name={coef 3} },
  columns={newcol,0,1,2},
  every head row/.style={before row=\toprule,after row=\midrule},
  every last row/.style={after row=\bottomrule},
]{CNR_opt_3b_3coefs_T2_80ms.txt}
\caption{Optimal contrast to noise ratio per coefficient, for a set of 3 shell, 
estimating 3 coefficients, assuming T2=80ms, per subject.
This assumes a total of 400 DW directions, and SNR in the $b=0$ image = 15.}
\end{table}

\begin{table}[htbp]
\centering
\pgfplotstabletypeset[
  fixed zerofill,
  precision=5,
  create on use/newcol/.style={ create col/set list={patient 1, patient 2, patient 3} },
  columns/newcol/.style={string type, column name={} },
  columns/0/.style={ column name={shell 1} },
  columns/1/.style={ column name={shell 2} },
  columns/2/.style={ column name={shell 3} },
  columns/3/.style={ column name={shell 4} },
  columns={newcol,0,1,2,3},
  every head row/.style={before row=\toprule,after row=\midrule},
  every last row/.style={after row=\bottomrule},
]{bvals_opt_4b_4coefs_T2_80ms.txt}
\caption{Optimal b-values for a set of 4 shell, estimating 4 coefficients, 
assuming T2=80ms, per subject.}
\end{table}

\begin{table}[htbp]
\centering
\pgfplotstabletypeset[
  fixed zerofill,
  precision=5,
  create on use/newcol/.style={ create col/set list={patient 1, patient 2, patient 3} },
  columns/newcol/.style={string type, column name={} },
  columns/0/.style={ column name={shell 1} },
  columns/1/.style={ column name={shell 2} },
  columns/2/.style={ column name={shell 3} },
  columns/3/.style={ column name={shell 4} },
  columns={newcol,0,1,2,3},
  every head row/.style={before row=\toprule,after row=\midrule},
  every last row/.style={after row=\bottomrule},
]{nDW_opt_4b_4coefs_T2_80ms.txt}
\caption{Optimal number of directions per shell for a set of 4 shell, 
estimating 4 coefficients, assuming T2=80ms, per subject.
This assumes a total of 400 DW directions, and SNR in the $b=0$ image = 15.}
\end{table}

\begin{table}[htbp]
\centering
\pgfplotstabletypeset[
  fixed zerofill,
  precision=5,
  create on use/newcol/.style={ create col/set list={patient 1, patient 2, patient 3} },
  columns/newcol/.style={string type, column name={} },
  columns/0/.style={ column name={coef 1} },
  columns/1/.style={ column name={coef 2} },
  columns/2/.style={ column name={coef 3} },
  columns/3/.style={ column name={coef 4} },
  columns={newcol,0,1,2,3},
  every head row/.style={before row=\toprule,after row=\midrule},
  every last row/.style={after row=\bottomrule},
]{CNR_opt_4b_4coefs_T2_80ms.txt}
\caption{Optimal contrast to noise ratio per coefficient, for a set of 4 shell, 
estimating 4 coefficients, assuming T2=80ms, per subject.
This assumes a total of 400 DW directions, and SNR in the $b=0$ image = 15.}
\end{table}

\begin{table}[htbp]
\centering
\pgfplotstabletypeset[
  fixed zerofill,
  precision=5,
  create on use/newcol/.style={ create col/set list={patient 1, patient 2, patient 3} },
  columns/newcol/.style={string type, column name={} },
  columns/0/.style={ column name={shell 1} },
  columns/1/.style={ column name={shell 2} },
  columns/2/.style={ column name={shell 3} },
  columns/3/.style={ column name={shell 4} },
  columns/4/.style={ column name={shell 5} },
  columns={newcol,0,1,2,3,4},
  every head row/.style={before row=\toprule,after row=\midrule},
  every last row/.style={after row=\bottomrule},
]{bvals_opt_5b_5coefs_T2_80ms.txt}
\caption{Optimal b-values for a set of 5 shell, estimating 5 coefficients, 
assuming T2=80ms, per subject.}
\end{table}

\begin{table}[htbp]
\centering
\pgfplotstabletypeset[
  fixed zerofill,
  precision=5,
  create on use/newcol/.style={ create col/set list={patient 1, patient 2, patient 3} },
  columns/newcol/.style={string type, column name={} },
  columns/0/.style={ column name={shell 1} },
  columns/1/.style={ column name={shell 2} },
  columns/2/.style={ column name={shell 3} },
  columns/3/.style={ column name={shell 4} },
  columns/4/.style={ column name={shell 5} },
  columns={newcol,0,1,2,3,4},
  every head row/.style={before row=\toprule,after row=\midrule},
  every last row/.style={after row=\bottomrule},
]{nDW_opt_5b_5coefs_T2_80ms.txt}
\caption{Optimal number of directions per shell for a set of 5 shells, 
estimating 5 coefficients, assuming T2=80ms, per subject.
This assumes a total of 400 DW directions, and SNR in the $b=0$ image = 15.}
\end{table}

\begin{table}[htbp]
\centering
\pgfplotstabletypeset[
  fixed zerofill,
  precision=5,
  create on use/newcol/.style={ create col/set list={patient 1, patient 2, patient 3} },
  columns/newcol/.style={string type, column name={} },
  columns/0/.style={ column name={coef 1} },
  columns/1/.style={ column name={coef 2} },
  columns/2/.style={ column name={coef 3} },
  columns/3/.style={ column name={coef 4} },
  columns/4/.style={ column name={coef 5} },
  columns={newcol,0,1,2,3,4},
  every head row/.style={before row=\toprule,after row=\midrule},
  every last row/.style={after row=\bottomrule},
]{CNR_opt_5b_5coefs_T2_80ms.txt}
\caption{Optimal contrast to noise ratio per coefficient, for a set of 5 shell, 
estimating 5 coefficients, assuming T2=80ms, per subject.
This assumes a total of 400 DW directions, and SNR in the $b=0$ image = 15.}
\end{table}

\end{document}



